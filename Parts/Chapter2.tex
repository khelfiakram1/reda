\chapter{Modélisation et calibrage de cameras}
\section{Introduction :}
Ce chapitre est consacré à la description du modèle qui correspond au processus de 
formation des images prises par une caméra. C'est pourquoi nous allons présenter le modèle 
sténopé qui peut nous permettre l'accès aux paramètres extrinsèques et intrinsèques de la 
caméra ainsi que les techniques utilisées pour son calibrage. Nous présenterons les relations 
ou les transformations homogènes entre les différents repères à savoir ; le repère caméra, le 
repère robot et le repère monde.
\section{Modélisation de la Caméra :}
\subsection{Conception}
Une caméra doit réaliser une transformation ponctuelle qui fait passer d’un point 
physique de l’espace réel 3D à un point 2D sur le plan image. Ce qui revient à une 
transformation mathématique de R3 vers R2.
Il existe plusieurs modèles dans la modélisation de la formation des images numériques. 
Notre étude prend comme modèle celui du sténopé ; appelé également le modèle perspectif 
(pin-hole en anglais) ; c'est un dispositif optique et le plus utilisé dans la vision par ordinateur. 
Ce modèle permet d'établir une relation entre un point de coordonnées 3D de la scène 
observée et sa projection dans l'image en 2D.
\subsection{ Composants d'une caméra :}
Une caméra se compose d'une boîte dont l'une de ses faces est percée d'un trou minuscule 
qui laisse entrer la lumière comme indique la figure (3.1). 
L'image vient se former sur la face opposée au trou et celle-ci peut être capturée en y 
plaçant un support photosensible (papier photographique).

image 

Une caméra perspective peut en effet être modélisée grâce au modèle du sténopé comme 
l'illustre la figure (3.2). Ce modèle associe à la caméra un repère cartésien ; ce repère dont son 
origine se situe sur le centre optique C est défini comme Rc = [Oc Xc Yc Zc ]. La caméra est 
représentée par un plan image, le plan image est parallèle aux axes Xc et Yc. Il est situé à une 
distance f de l'origine C appelée distance focale f de ce plan. La droite passant par le centre 
de projection et perpendiculaire au plan image est L'axe optique. L'intersection de l'axe 
optique avec le plan image est appelé point principal.

image

\section{Calibration D'une Caméra :}
Le calibrage consiste à estimer les paramètres intrinsèques et extrinsèques d'un modèle de 
caméra à partir d'un ensemble de points 3-D et de leur image. Il s'agit donc d'estimer les 
éléments de la matrice (3.7) ; Cette étape est incontournable pour de nombreuses applications 
de vision par l’ordinateur.Le calibrage de la caméra a été préalablement traité par la communauté de la 
photogrammétrie. Par conséquent, de nombreuses méthodes de calibrage ont été proposées
dans la littérature. Ces approches sont généralement classifiées en deux catégories :
\subsection{Calibrage avec un objet 3D de référence ou une mire :}
Cette technique utilise l’observation d’objets en 3D avec des coordonnées connues. Les 
objets de calibrage (mire) (figure (3.7)) sont généralement des points répartis sur des plans 
orthogonaux ou sur un plan translaté dans la direction de sa normale. Le calcul peut alors être 
effectué de façon relativement simple [dib11].
\subsection{Calibrage automatique (ou auto-calibrage) :}
Dans cette technique, Le mouvement connu de la caméra filmant une scène statique est 
utilisé pour poser des contraintes sur les paramètres intrinsèques prenant en compte la rigidité 
des objets filmés en utilisant uniquement les informations de l’image [dib11].
Nous avons opté pour la première famille ; sachant qu'elle est disponible sur internet " 
camera calibration toolbox for matlab". Elle consiste à calibrer la caméra à partir de 
plusieurs images d'une mire. Prises sous des points de vue différents comme indique la figure
(3.8).

image1
image2

Les positions des coins de chaque carré de la mire sont alors extraites puis raffinées en 
cas de distorsion. Ensuite les paramètres de la caméra sont déterminés par optimisation non 
linéaire.
\section{Résultat de la calibration de la caméra :}
Nous avons utilisé une webcam, type Logitech 720 hp (figure (3.9)) pour avoir un 
modèle et procéder à l'expérimentation. La caméra est placée sur un robot mobile qui se 
trouve au niveau du département d'électronique pour déterminer les paramètres intrinsèques 
de la caméra obtenus lors de la phase de calibrage. Leurs valeurs sont résumées dans le 
tableau (3.1).

image

tableau

La figure suivante présente le repère de la caméra Rc(Oc,Xc,Yc,Zc). La pyramide 
rouge correspond au champ de vue effectif de la caméra défini par le plan d'image.

image

Sur cette nouvelle figure, chaque position et orientation de la caméra sont représentées
par une pyramide verte par rapport à chaque image de calibration.
Estimation relative de la pose d'une caméra par rapport à chaque image de 
calibration.
%http://eprints.univ-batna2.dz/1125/1/ing%20Bouhamatou%20Zoulikha.pdf

image 

\section{ Définition des repères : }
La modélisation du robot s'appuie sur différents repères Figure (3.11) définis comme suit :
R(O,XY,Y: Le repère lié à la scène repère monde)
Rr(Or,Xr,Yr,Yr): Le repère lié à la base mobile
Rc(Oc,Xc,Yc,Yc): Le repère lié à la caméra

Pour la base mobile ; la position est représentée par les coordonnées (x,y) du point, Or dans le repère Rr tandis que l'orientation est donnée par l'angle T .la position de la caméra dans le repère est décrite par le vecteur Oc Or=(T,O,H).
Qb(X0,Y0,Z0,1): Un point de trajectoire est exprimé dans le repère monde. Sa projection 
dans le plan image est Q1(U1,V1,1) Alors, comme nous l'avons vu la relation :Q1=K*Tcb*Qb
Après calibrage, la matrice K est déterminée et le problème consiste ensuite à trouver la 
Tcb
La relation entre le point de l'espace dans le repère monde et repère caméra est :
Qc=Tcr*Trb*Qb
\section{Résultats de Simulations et discussion :}
Pour effectuer un asservissement visuel avec les paramètres obtenus, Nous avons 
procéder à une manipulation qui consiste à fournir au robot mobile les coordonnées 
échantillonnées d'une trajectoire que nous avons tracé sur une surface bien illuminée. Pour 
vérifier l'exactitude des coordonnées mesurées, nous avons choisi deux points sur la 
trajectoire à poursuivre en calculant leurs valeurs et les comparer avec celles mesurées (figure 
(3.13)). Les résultats de comparaison sont très satisfaisants et l'erreur est presque inexistante. 
 - la position de la caméra par rapport au repère monde est : 
[X,Y,Z]pow(r) = [20cm, 25cm, 30 cm]pow(T)
 -L'angle de rotation a été choisi entre le repère caméra et le repère monde.

 image tableau 
 \section{.Conclusion :}
 Après avoir présenté la partie calibrage et modélisation de la caméra nous avons procédé 
au calibrage d'une Webcam caméra que nous avons choisi en tenant compte du prix et de la 
qualité. Cette étape est nécessaire pour déterminer les paramètres intrinsèques et extrinsèques 
de la caméra. La modélisation de la caméra présente la projection du point de l'espace 3D à un 
point 2D sur le plan image par trois transformations élémentaire successives : la 
transformation rigide, la projection perspective et la transformation affine.
